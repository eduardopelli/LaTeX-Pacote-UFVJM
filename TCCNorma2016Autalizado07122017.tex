\documentclass[12pt,a4paper,twoside,sumario=abnt-6027-2012,chapter=TITLE]{abntex2}

\usepackage{scrextend} % para usar o ambiente labeling
\usepackage{array}


\setlength\afterchapskip{\lineskip} 
\setlength\beforechapskip{\lineskip}
%coloca o espaçamento 1,5 entre títulos de capítulos e o texto. Coloca tanto antes quanto depois. Norma abnt e manual da UFVJM

\setlength\aftersecskip{\lineskip}
\setlength\beforesecskip{\lineskip}
%coloca o espaçamento 1,5 entre títulos de seções e o texto. Coloca tanto antes quanto depois. Norma abnt e manual da UFVJM

\setlength\aftersubsecskip{\lineskip}
\setlength\beforesubsecskip{\lineskip}
%coloca o espaçamento 1,5 entre títulos de subseções e o texto. Coloca tanto antes quanto depois. Norma abnt e manual da UFVJM

\setlength\aftersubsubsecskip{\lineskip}
\setlength\beforesubsubsecskip{\lineskip}
%coloca o espaçamento 1,5 entre títulos de subsubseções e o texto. Coloca tanto antes quanto depois. Norma abnt e manual da UFVJM

\usepackage[brazil]{babel}
%\usepackage[brazil]{babel}
\usepackage{mathptmx} %Times new roman para texto e matemática
%\usepackage{times} %Times new roman para texto
\usepackage{lastpage}
\usepackage{minitoc} 
\usepackage[utf8]{inputenc}
\usepackage{amsmath}
\usepackage{amsfonts}
\usepackage{amssymb}
\usepackage{graphicx}	

%%% Euler - pacotes para múltiplas linhas e colunas de tabelas
\usepackage{multirow}
\usepackage{multicol}

%%%%%%

%%%marcelo - Caption do tipo do elemento gráfico em negrito.

%\usepackage[labelfont=bf]{caption}

%%% Ex: Figura 1 - Título da figura.
%%% Apenas Figura 1 estará em negrito

%%%marcelo - Todo caption em negrido

\usepackage{caption}
% marcelo - colocar a fonte alinhada a esquerda
%https://tex.stackexchange.com/questions/131532/position-caption-of-centered--left-adjusted-to-figure
%\captionsetup{font={bf,small}, position=below, justification=raggedright, singlelinecheck=false} % negrito e menor
\captionsetup{font={bf,small}, position=below} % negrito e menor
%%% tamanhos disponíveis -> scriptsize, footnotesize, small, normalsize, large, Large

%%% Ex: Figura 1 - Título da figura
%%% Figura 1 - Título da figura estará em negrito.

% marcelo - Deixa a fonte da figura, quadro, etc, sem negrito quando se utiliza a solução acima.
%\newcommand{\fonteUFVJM}[1]{\legend{\textnormal{Fonte -- #1}}}
\renewcommand{\fonte}[1]{\captionsetup{justification=raggedright, singlelinecheck=false}\legend{\textnormal{Fonte: #1}}}

%\renewcommand{\fonte}[1]{\legend{\textnormal{Fonte: #1}}}


% marcelo - simula aspas duplas no estilo usado pelo JSON (''a'')
\newcommand{\q}[1]{{\textquotesingle\textquotesingle}#1{\textquotesingle\textquotesingle}}
%marcelo - nome menor para backslash
\newcommand{\bks}{\textbackslash}

\usepackage{tabularx} % marcelo - tabelas justificadas
\usepackage{booktabs} % marcelo - grossura da linha em tabelas

%%% marcelo - retira espaços extras entre os elementos de uma lista (itemize e enumarate)
\usepackage{enumitem}
\usepackage{setspace}
\setlist[itemize,enumerate]{nosep} % Caso queira deixar espaçamento duplo entre o parágravo anterior e posterior, mas não entre itens da lista, utilizar noitemsep no lugar de nosep
%%% até aqui

\raggedbottom % marcelo - Resolve o problema do espaçamento entre parágrafos aleatório

\usepackage{subfig} %marcelo - Suporte a subfiguras	

%\usepackage[none]{hyphenat} %marcelo - não permite hifenização


\usepackage{indentfirst}		
\usepackage{color}				
\usepackage{microtype}
%\usepackage[alf,abnt-emphasize=bf]{abntex2cite}

%resolve problemas com as referências
\usepackage[alf,abnt-etal-cite=3,abnt-etal-list=0,abnt-etal-text=emph,abnt-emphasize=bf,bibjustif]{abntex2cite}

\usepackage{ragged2e} %para usar o comando \justify
\usepackage{enumitem} %para mudar as alíneas
\usepackage{textcomp} %fornece caracteres especiais
\setlength{\parindent}{2cm} %Tamanho do parágrafo
%\setlength{\parskip}{1.5cm} 
%\setlrmarginsandblock{3cm}{2cm}


%Para definir o tamanho das fontes das seções, subseções, etc (+++ seria o item (subsection, section, chapter, etc):

\renewcommand{\ABNTEXchapterfontsize}{\normalsize} 
\renewcommand{\ABNTEXsectionfontsize}{\normalsize} 
\renewcommand{\ABNTEXsubsectionfontsize}{\normalsize} 
\renewcommand{\ABNTEXsubsubsectionfontsize}{\normalsize}

%Para deixas as fontes das seções, subseções, etc, em negrito (+++ seria o item (subsection, section, chapter, etc):

\renewcommand{\ABNTEXchapterfont}{\normalfont\normalfont\bfseries}
\renewcommand{\ABNTEXsectionfont}{\normalfont\bfseries}
\renewcommand{\ABNTEXsubsectionfont}{\normalfont\itshape\bfseries}
\renewcommand{\ABNTEXsubsubsectionfont}{\normalfont\itshape}


%Alterar o tipo da fonte no sumário - MARCELO

% Secao secundaria (Section) Caixa baixa, Negrito, tamanho 12
%\renewcommand{\cftsectionfont}{\bfseries} %ponha \rmfamily se quiser serifadas...

% Secao terciaria (Subsection) Caixa baixa e italico no sumário
\renewcommand{\cftsubsectionfont}{\normalfont\itshape\bfseries}

% Secao quaternaria (Subsubsection) Caixa baixa, Negrito, sublinhado, tamanho 12
\renewcommand{\cftsubsubsectionfont}{\normalfont\itshape}

% Seção quinaria (subsubsubsection) Caixa baixa, sem negrito, tamanho 12
%\renewcommand{\cftparagraphfont}{\normalfont}

% MARCELO - Evitar linhas órfãs e viúvas
\widowpenalty=10000
\clubpenalty=10000



%%%%% Euler - Cria comando para citar fontes de figuras e tabelas.

\newcommand{\citefonte}[1]
{\citeauthor{#1}, \citeyear{#1}}


\hypersetup{
     	pagebackref=true,
		pdftitle={\@title}, 
		pdfauthor={\@author},
    	pdfsubject={\imprimirpreambulo},
	    pdfcreator={Nome do Aluno},
		pdfkeywords={abnt}{latex}{abntex}{abntex2}{trabalho acadêmico}, 
		colorlinks=true,       		
    	linkcolor=black,          	
    	citecolor=black,        	
    	filecolor=magenta,      		
		urlcolor=blue,
		bookmarksdepth=4
}
\makeatother
\makeindex

\titulo{TÍTULO DO TRABALHO: subtítulo}
\autor{Euler Guimarães Horta}
\local{Diamantina}
\data{2018}
\orientador{Antônio de Pádua Braga} % (não incluir “Prof.” ou titulação)
\coorientador {René Natowicz} %  (se houver)(não incluir “Prof.” ou titulação)
\tipotrabalho{Dissertação de Mestrado}
\preambulo{Dissertação de Mestrado apresentada ao Programa de Pós-Graduação em Educação, como parte dos requisitos exigidos para a obtenção do título de Mestre em Educação.}



\begin{document}
\frenchspacing 

\renewcommand{\imprimircapa}{% 
\begin{capa}% 
\center 
\textbf{UNIVERSIDADE FEDERAL DOS VALES DO JEQUITINHONHA E MUCURI}\\
{\textbf{Programa de Pós-Graduação em Educação}}\\
{\bfseries \imprimirautor} 
%\vspace*{1cm} 
\vfill 
\begin{center} 
\bfseries \imprimirtitulo 
\end{center} 
\vfill 

\bfseries \imprimirlocal 

\bfseries \imprimirdata 
\vspace*{1cm} 
\end{capa} }

\imprimircapa

\makeatletter
\renewcommand
{\folhaderostocontent}{ 
\begin{center} 
{\bfseries\imprimirautor} 
\vspace*{\fill}\vspace*{\fill} 
\begin{center} 
\bfseries\imprimirtitulo 
\end{center} 
\vspace*{\fill} 
\abntex@ifnotempty{\imprimirpreambulo}{% 
\hspace{.45\textwidth} 
\begin{minipage}{.5\textwidth} 
\begin{SingleSpacing}
\imprimirpreambulo
\end{SingleSpacing}
%\vspace{1.5cm}

\begin{SingleSpacing}
{\imprimirorientadorRotulo~\imprimirorientador}\\ 
\abntex@ifnotempty{\imprimircoorientador}{% 
{\imprimircoorientadorRotulo~\imprimircoorientador}% 
}% 
\end{SingleSpacing}
\end{minipage}%
 
\vspace*{\fill} }% 
{\abntex@ifnotempty{\imprimirinstituicao}{\imprimirinstituicao \vspace*{\fill}}} 

\vspace*{\fill} {\bfseries\imprimirlocal} 
\par 
{\bfseries\imprimirdata} 
\vspace*{1cm} 
\end{center} 
} 
\makeatother

\pagenumbering{arabic}
\imprimirfolhaderosto

\begin{folhadeaprovacao}

  \begin{center}
    {\bfseries \imprimirautor}

    \vspace{1.5cm}
    \begin{center}
      \bfseries \imprimirtitulo
    \end{center}
    \vspace*{\fill}
    
    \hspace{.45\textwidth}
    \begin{minipage}{.5\textwidth}
        \imprimirpreambulo\\
        
        {Orientador: Prof. Dr. \imprimirorientador}\\ % (incluir Prof./Profa. e titulação, Dr./Dra./Me.)
        {Coorientador: Prof. Dr. \imprimircoorientador}\\ % (incluir Prof./Profa. e titulação, Dr./Dra./Me.)
        
        {Data de aprovação \_\_\_\_/\_\_\_\_/\_\_\_\_.}
    \end{minipage}
    \vspace*{\fill}
   \end{center}
  


   \assinatura{Prof. Dr. \imprimirorientador \\ Departamento de Eletrônica - UFMG} 
   \assinatura{Prof. Dr. \imprimircoorientador \\ Instituto de Ciência e Tecnologia - UFVJM} 
   \assinatura{Prof. Dr. Alexandre Ramos Fonseca  \\ Instituto de Ciência e Tecnologia - UFVJM}
   \assinatura{Prof. Dr. Luiz Antônio Aguirre \\ Departamento de Eletrônica - UFMG}
  
      
   \begin{center}
    \vspace*{0.5cm}
    {\textbf{\imprimirlocal}}
    %\par
    %{\imprimirdata}
    %\vspace*{1cm}
  \end{center}

\end{folhadeaprovacao}


%%%%%%%%%%%%%% Dedicatória - Opcional

%\begin{dedicatoria}
%\vspace*{\fill}
%\begin{flushright}
%Na dedicatória, o autor presta uma homenagem ou dedica seu trabalho a alguém. Essa é apenas uma sugestão para a dedicatória.
%\end{flushright}
%\end{dedicatoria}

%%%%%%%%%%%%%% Agradecimentos

\begin{agradecimentos}
O conteúdo dessa seção é livre e normalmente é dedicado a agradecer as pessoas que contribuíram com o trabalho.
\end{agradecimentos}

%%%%%%%%%%%%%% Epígrafe - Opcional

% \begin{epigrafe}
% \vspace*{\fill}
% \begin{flushright}
% \justify
% Essa é uma sugestão de formatação de Epígrafe. O texto sempre deve estar relacionado ao tema do trabalho ou capítulo. 
% Caso o trabalho possua epígrafe, esta deverá ser referenciada, como neste exemplo \cite{CHAMPY2010}.%(CHAMPY, 2010, p. 117).
% \end{flushright}
% \end{epigrafe}
 
%%%%%%%%%%%%%% Resumo
 
\input{Resumo}
\cleardoublepage
%%%%%%%%%%%%%% Abstract

\input{Abstract}
\cleardoublepage

%%%%%%%%%%%%%% Resumo

\pdfbookmark[0]{\listfigurename}{lof}
\listoffigures*
\cleardoublepage

\pdfbookmark[0]{\listtablename}{lot}
\listoftables*
\cleardoublepage


\begin{siglas}
\item UFVJM - Universidade Federal dos Vales do Jequitinhonha e Mucuri
\item UFMG - Universidade Federal de Minas Gerais
\item IFNMG - Instituto Federal de Educação, Ciência e Tecnologia do Norte de Minas Gerais
\end{siglas}


\newpage

\begin{simbolos}
\item[$ \Gamma $] Letra grega Gama
\item[$ \Lambda $] Lambda
\item[$ \zeta $] Letra grega minúscula zeta
\item[$ \in $] Pertence
\end{simbolos}

\newpage

\tableofcontents*

\newpage
\textual
\pagestyle{simple} %%%% Esse comando tira o cabeçalho do modelo abntex2 original.

\input{Introducao}
\input{Revisao}
\input{ExemplosDeUso}
\input{MateriaisEMetodos}
%\input{ResultadosEDiscussoes}
%\input
% Deixar espaço em branco entre o final do trabalho e a bibliografia para evitar problemas de espaçamento entre linhas
%%%%%%%%%%%%%%%%%

%%%%%%%%%%%%%%%%%
\bibliography{bibliografia}
\end{document}


