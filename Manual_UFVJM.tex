\documentclass[
	% -- opções da classe memoir --
	12pt,				% tamanho da fonte
	openright,			% capítulos começam em pág ímpar (insere página vazia caso preciso)
	twoside,			% para impressão em recto e verso. Oposto a oneside
	a4paper,			% tamanho do papel. 
	% -- opções da classe abntex2 --
	chapter=TITLE,		% títulos de capítulos convertidos em letras maiúsculas
	%section=TITLE,		% títulos de seções convertidos em letras maiúsculas
	%subsection=TITLE,	% títulos de subseções convertidos em letras maiúsculas
	%subsubsection=TITLE,% títulos de subsubseções convertidos em letras maiúsculas
	% -- opções do pacote babel --
%	sumario=tradicional,
	sumario=abnt-6027-2012,
	english,			% idioma adicional para hifenização
	brazil				% o último idioma é o principal do documento
	]{UFVJM-abntex2}
	%]{article}

%\usepackage{CUSTOMIZACOES}

\usepackage[left=3cm,right=2cm,top=3cm,bottom=2cm]{geometry}

\setlrmarginsandblock{3cm}{2cm}{*}
\setulmarginsandblock{3cm}{2cm}{*}
\checkandfixthelayout
% ---
% Pacotes básicos 
% ---

%\usepackage{lmodern}			% Usa a fonte Latin Modern

%Essa times esta obsoleta
%\usepackage{times}			% Usa a fonte Times		
\usepackage{mathptmx}           % Usa a fonte times
\renewcommand{\ABNTEXchapterfont}{\rmfamily\bfseries}



%\renewcommand{\ABNTEXchapterfont}{\normalfont\bfseries} %bold chapter
%Teste
% \renewcommand*{\ABNTEXchapterleader}{
% \normalfont\ABNTEXdotfill{\ABNTEXsectiondotsep}}

% Travessão no sumário para apêndices e anexos \\ 
%\renewcommand{\cftchapteraftersnum}{\hfill\textendash\hfill} \\
% % Insere espaço entre os itens do sumário de diferentes capítulos \\
% \setlength{\cftbeforechapterskip}{1.0em \@plus\p@}

%\setallmainfonts{Times New Roman}

%Era assim
%\usepackage[T1]{fontenc}		% Selecao de codigos de fonte.
%\usepackage[utf8]{inputenc}		% Codificacao do documento (conversão automática dos acentos)
%\usepackage[utf8x]{inputenc}
\usepackage{lastpage}			% Usado pela Ficha catalográfica
%\usepackage{indentfirst}		% Indenta o primeiro parágrafo de cada seção.
%\usepackage{color}				% Controle das cores
%\usepackage{graphicx}			% Inclusão de gráficos
%\usepackage{microtype} 			% para melhorias de justificação
% ---
%\usepackage{paralist}			% para usar compactitem
\usepackage[final]{pdfpages}		% para incluir pdf

\usepackage[T1]{fontenc}		% Selecao de codigos de fonte.
\usepackage[utf8]{inputenc}		% Codificacao do documento (conversão automática dos acentos)


\usepackage{indentfirst}		% Indenta o primeiro parágrafo de cada seção.
\usepackage{nomencl} 			% Lista de simbolos
\usepackage{color}				% Controle das cores
\usepackage{graphicx}			% Inclusão de gráficos
\usepackage[export]{adjustbox}

\usepackage{microtype} 			% para melhorias de justificação
\usepackage{amsmath}			% para tratar o case no ambiente math

\usepackage{enumitem}

\usepackage{multirow}
% ---

 %cores nas tabelas%
\usepackage{colortbl}
 


%\usepackage{titlecaps}% http://ctan.org/pkg/titlecaps
\usepackage{titlesec}
%\usepackage[compact]{titlesec}

%\titleformat{\chapter}{\normalfont\LARGE\bfseries}{\MakeUppercase{\thechapter}}{.5em}{\vspace{.5ex}}[\titlerule]
\titleformat{\section}
  {\normalfont\bfseries}
  {\thesection}
  {5pt}
  {\MakeUppercase}
  
  \titleformat{\subsection}
  {\normalfont\bfseries}
  {\thesubsection}
  {5pt}
  {}
  
  \titleformat{\chapter}
  {\centering\normalfont\bfseries}
  {\thechapter}
  {5pt}
  {\MakeUppercase}
  
%\setallmainfonts{Times New Roman}
%\setmainfont{Times New Roman}

\usepackage{tikz}
\usetikzlibrary{arrows,calc,patterns,decorations.markings} 
\usetikzlibrary{decorations.text}
\usetikzlibrary{arrows.meta, calc, quotes}

%\usetikzlibrary{arrows.meta, calc, quotes}

\usepackage[portuguese, ruled, linesnumbered]{algorithm2e} %Para criar algoritmos

\usepackage{eso-pic}			% colocar imagem da UFVJM na capa
	

%\DeclareUnicodeCharacter{00A0}{~}	
% ---
% Pacotes adicionais, usados apenas no âmbito do Modelo Canônico do abnteX2
% ---
\usepackage{lipsum}				% para geração de dummy text
% ---

% Titulos dos capitulos e secoes
%\usepackage{titlesec}
%\usepackage{titling}

%\usepackage{fontspec}
%\newfontfamily{\headingfont}{Times New Roman}
% %Specify different font for section headings
%\makeheadstyles{myheadings}{%
%\setsecheadstyle{\headingfont\bfseries} 
%\setsubsecheadstyle{\headingfont\itshape}
%}
%
%\headstyles{myheadings} 

% Specify different font for section headings
%\usepackage{titlesec}
%\usepackage{titling}
%\usepackage{fontspec}
% Specify different font for section headings
%\newfontfamily\headingfont[]{Times New Roman}
%\titleformat*{\section}{\LARGE\headingfont}
%\titleformat*{\subsection}{\Large\headingfont}
%\titleformat*{\subsubsection}{\large\headingfont}
%\renewcommand{\maketitlehooka}{\headingfont}

%\usepackage{subfig}
%
%\usepackage{enumerate}
%\usepackage{enumitem}

%\usetikzlibrary{shapes.gates.logic.US,trees,positioning,arrows}
%\usetikzlibrary{arrows,shapes,positioning,shadows,trees}



%\usepackage{color}

\definecolor{pblue}{rgb}{0.13,0.13,1}
\definecolor{pgreen}{rgb}{0,0.5,0}
\definecolor{pred}{rgb}{0.9,0,0}
\definecolor{pgrey}{rgb}{0.46,0.45,0.48}

\usepackage{listings}
%\lstset{language=Java,
%  showspaces=false,
%  showtabs=false,
%  breaklines=true,
%  showstringspaces=false,
%  breakatwhitespace=true,
%  commentstyle=\color{pgreen},
%  keywordstyle=\color{pblue},
%  stringstyle=\color{pred},
%  basicstyle=\ttfamily,
%  moredelim=[il][\textcolor{pgrey}]{$$},
%  moredelim=[is][\textcolor{pgrey}]{\%\%}{\%\%}
%}
\newcommand{\li}{\textit{LiDAR }}
\newcommand{\kps}{\textit{keypoints} }

\newcommand{\cA}{5}
\newcommand{\cB}{1}
\newcommand{\cC}{0.5}

\titlespacing*{\chapter}
{0pt}{.75cm}{.75cm}
\titlespacing*{\section}
{0pt}{.75cm}{.75cm}

\titlespacing*{\subsection}
{0pt}{.75cm}{.75cm}



\renewcommand\thesection{\arabic{section}}
\renewcommand{\theequation}{\arabic{equation}}
%\renewcommand{\ABNTEXpartname}{Artigo}
%\def\partname{Artigo}

%\captionstyle{\raggedright} %(para as legendas; use \legend para fonte)

% Fontes padrao de part, chapter, section, subsection e subsubsection
\renewcommand{\ABNTEXchapterfont}{\rmfamily}
\renewcommand{\ABNTEXchapterfontsize}{\normalsize\bfseries}

\renewcommand{\ABNTEXsectionfont}{\normalsize\bfseries}

\renewcommand{\ABNTEXpartfont}{\ABNTEXchapterfont}
\renewcommand{\ABNTEXpartfontsize}{\ABNTEXchapterfontsize}


% Titulos das Figuras e tabelas
\captionnamefont{\fontsize{10pt}{\baselineskip}\bfseries}
\captiontitlefont{\fontsize{10pt}{\baselineskip}\bfseries\centering}


%Espaçode titulos
\titlespacing{\chapter}{0pt}{-14mm}{.75cm}

% Espaco de fonte das figurs e tabelas
% Define o comando \fonte que respeita as configurações de caption do memoir ou do caption
\newcommand{\Fonte}[2]{%
\begin{flushleft} 
\vspace{0cm}
\hspace{#1}
    \begin{minipage}{6cm}
   \footnotesize Fonte: #2
    \end{minipage}%
\end{flushleft} 
}


\lstset{ %
  backgroundcolor=\color{white}, 
  basicstyle=\footnotesize,       
  breakatwhitespace=false,        
  breaklines=true,                 
  captionpos=b,                    
  commentstyle=\color{pgreen},   
  escapeinside={\%*}{*)},        
  extendedchars=true,              
  frame=single,                  
  keywordstyle=\color{blue},       
  language=Java,                
  numbers=left,                    
  numbersep=5pt,                   
  numberstyle=\tiny\color{pgray},
  rulecolor=\color{black},        
  showspaces=false,               
  showstringspaces=false,          
  showtabs=false,                  
  stepnumber=2,                    
  stringstyle=\color{mymauve},   
  tabsize=2,                      
  title=\lstname, 
  xleftmargin=\parindent,
%  xleftmargin=.25in,
%  xrightmargin=.25in,
  morekeywords={not,\},\{,preconditions,effects },            
  deletekeywords={time}            
}

\usepackage{hyphenat}
\hyphenation{Mu-cu-ri}

% ---
% Pacotes de citações
% ---
\usepackage[brazilian,hyperpageref]{backref}	 % Paginas com as citações na bibl
\usepackage[alf]{abntex2cite}	% Citações padrão ABNT

%\usepackage{fancyhdr}
%\pagestyle{myheadings}
%\pagestyle{myheadings}
%\pagestyle{fancy}
%\pagestyle{fancyplain}
%\renewcommand{\headrulewidth}{0pt}
%\renewcommand{\footrulewidth}{0pt}
%\fancyhf{}
%\fancyhead{} % clear header
%\fancyfoot{} % clear footer
%\fancyfoot[LE,RO]{\thepage} % page at left on even and right on odd pages

%\pagestyle{fancy}
%\pagestyle{fancy}

%\fancypagestyle{myheadings}{ %
 % \fancyhf{} % remove everything
 % \fancyhead[LE,RO]{\thepage}% clear header
%\fancyfoot{} % clear footer
 % \renewcommand{\headrulewidth}{0pt} % remove lines as well
  %\renewcommand{\footrulewidth}{0pt}
%}



\pagestyle{myheadings}
\makeheadrule{abntheadings}{\textwidth}{0pt}
% --- 
% CONFIGURAÇÕES DE PACOTES
% --- 

% ---
% Configurações do pacote backref
% Usado sem a opção hyperpageref de backref
\renewcommand{\backrefpagesname}{Citado na(s) página(s):~}
% Texto padrão antes do número das páginas
\renewcommand{\backref}{}
% Define os textos da citação
\renewcommand*{\backrefalt}[4]{
	\ifcase #1 %
		Nenhuma citação no texto.%
	\or
		Citado na página #2.%
	\else
		Citado #1 vezes nas páginas #2.%
	\fi}%
% ---

% ---
% Informações de dados para CAPA e FOLHA DE ROSTO
% ---
\titulo{\normalsize Mapeamento do potencial madeireiro em área de floresta amazônica}
\autor{Eduardo Pelli}
\local{\normalsize Diamantina}
\data{\normalsize 2019}
\orientador{\normalsize Prof. Dr. Eric Bastos G\"orgens}
%\coorientador{Euler Guimar\~aes Horta}
\instituicao{%
  \normalsize Universidade Federal dos Vales do Jequitinhonha e Mucuri - UFVJM
  \par
  Departamento de Engenharia Florestal - DEF
  \par
  Programa de P\'os-Gradua\c c\~ao em Ci\^encia Florestal - PPGCF}
\tipotrabalho{Tese (Doutorado)}
% O preambulo deve conter o tipo do trabalho, o objetivo, 
% o nome da instituição e a área de concentração 
%\preambulo{Modelo canônico de trabalho monográfico acadêmico em conformidade com
%as normas ABNT apresentado à comunidade de usuários \LaTeX.}
\preambulo{Tese de Doutorado apresentada ao programa de Pós-graduação em Ciência Florestal da Universidade Federal dos Vales do Jequitinhonha e Mucuri, como requisito para obtenção do título de Doutor.}
%Tese de doutorado apresentada como requisito parcial para a obtenção do título de doutor em Ciência Florestal.
% ---


% ---
% Configurações de aparência do PDF final

% alterando o aspecto da cor azul
\definecolor{blue}{RGB}{41,5,195}

% informações do PDF
\makeatletter
\hypersetup{
     	%pagebackref=true,
		pdftitle={\@title}, 
		pdfauthor={\@author},
    	pdfsubject={\imprimirpreambulo},
	    pdfcreator={LaTeX with abnTeX2},
		pdfkeywords={abnt}{latex}{abntex}{abntex2}{trabalho acadêmico}, 
		colorlinks=true,       		% false: boxed links; true: colored links
    	linkcolor=blue,          	% color of internal links
    	citecolor=blue,        		% color of links to bibliography
    	filecolor=magenta,      		% color of file links
		urlcolor=blue,
		bookmarksdepth=4
}
\makeatother
% --- 

\makeatletter
\renewenvironment{abstract}{%
    \if@twocolumn
      \section*{\abstractname}%
    \else %% <- here I've removed \small
      \begin{center}%
        {\bfseries \normalsize\abstractname\vspace{\z@}}%  %% <- here I've added \Large
      \end{center}%
     % \quotation
    \fi}
    {\if@twocolumn\else\fi}
%    {\if@twocolumn\else\endquotation\fi}
\makeatother


% --- 
% Espaçamentos entre linhas e parágrafos 
% --- 

% O tamanho do parágrafo é dado por:
\setlength{\parindent}{2.0cm}
%\setlength\parindent{2cm}

% Controle do espaçamento entre um parágrafo e outro:
\setlength{\parskip}{0.cm}  % tente também \onelineskip

% ---
% compila o indice
% ---
\makeindex

% ---

% ----
% Início do documento
% ----
\begin{document}
\linespread{1.25}
% Seleciona o idioma do documento (conforme pacotes do babel)
%\selectlanguage{english}
\selectlanguage{brazil}

% Retira espaço extra obsoleto entre as frases.
\frenchspacing 

% ----------------------------------------------------------
% ELEMENTOS PRÉ-TEXTUAIS
% ----------------------------------------------------------
% \pretextual

% Posicao da watermark da capa
\AddToShipoutPicture*{\put(-610,0){\includegraphics{img/capa.eps}}}
%\AddToShipoutPicture*{\put(0,0){\includegraphics{img/capa.eps}}}
% ---
% Capa
%\begin{center}
%\normalsize
%\textbf{UNIVERSIDADE FEDERAL DOS VALES DO JEQUITINHONHA E MUCURI}\\
%\vspace{-.2cm}
%\textbf{Programa de Pós-Graduação em Ciência Florestal}\\
%\vspace{.3cm}
%\textbf{Eduardo Pelli}\\
%\vspace{8cm}
%\textbf{\uppercase{Mapeamento do potencial madeireiro em área de floresta amazônica}}\\
%\vfill
%\textbf{Diamantina}\\
%\vspace{-.2cm}
%\textbf{2019}\\
%
%\end{center}
%\newpage
% ---
\imprimircapa
% ---

% ---
% Folha de rosto
% (o * indica que haverá a ficha bibliográfica)
% ---
\imprimirfolhaderosto*
% ---
% Inserir a ficha bibliografica
% ---
% Isto é um exemplo de Ficha Catalográfica, ou ``Dados internacionais de
% catalogação-na-publicação''. Você pode utilizar este modelo como referência. 
% Porém, provavelmente a biblioteca da sua universidade lhe fornecerá um PDF
% com a ficha catalográfica definitiva após a defesa do trabalho. Quando estiver
% com o documento, salve-o como PDF no diretório do seu projeto e substitua todo
% o conteúdo de implementação deste arquivo pelo comando abaixo:
%

%\begin{fichacatalografica}
%\includepdf{fig_ficha_catalografica.pdf}
%\includepdf{folhadeaprovacao_final.pdf}
%\end{fichacatalografica}

%\begin{fichacatalografica}
%	%\sffamily
%	\small
%	\vspace*{\fill}					% Posição vertical
%	\begin{center}					% Minipage Centralizado
%	\fbox{\begin{minipage}[c][7.5cm]{14.4cm}		% Largura
%	\small
%	\imprimirautor
%	%Sobrenome, Nome do autor
%	
%	\hspace{0.5cm} \imprimirtitulo  / \imprimirautor. 
%	\imprimirlocal, \imprimirdata.
%	
%	\hspace{0.5cm} \pageref{LastPage} p. : il; 30 cm.\\
%	
%	\hspace{0.5cm} \imprimirorientadorRotulo~\imprimirorientador\\
%	
%	\hspace{0.5cm}
%	\parbox[t]{\textwidth}{\imprimirtipotrabalho~--~\\\imprimirinstituicao,
%	\imprimirdata.}\\
%	
%	\hspace{0.5cm}
%		1. LiDAR.
%		2. Visão computacional.
%		3. Manejo florestal de precisão.
%		I. Orientador.
%		II. Universidade Federal dos Vales do Jequitinhonha e Mucuri - UFVJM.
%		III.  Departamento de Engenharia Florestal - DEF
%		IV. \imprimirtitulo		
%	\end{minipage}}
%	\end{center}
%\end{fichacatalografica}
% ---

% ---
% Inserir errata
% ---
%\begin{errata}
%Elemento opcional da \citeonline[4.2.1.2]{NBR14724:2011}. Exemplo:
%
%\vspace{\onelineskip}
%
%FERRIGNO, C. R. A. \textbf{Tratamento de neoplasias ósseas apendiculares com
%reimplantação de enxerto ósseo autólogo autoclavado associado ao plasma
%rico em plaquetas}: estudo crítico na cirurgia de preservação de membro em
%cães. 2011. 128 f. Tese (Livre-Docência) - Faculdade de Medicina Veterinária e
%Zootecnia, Universidade de São Paulo, São Paulo, 2011.
%
%\begin{table}[htb]
%\center
%\footnotesize
%\begin{tabular}{|p{1.4cm}|p{1cm}|p{3cm}|p{3cm}|}
%  \hline
%   \textbf{Folha} & \textbf{Linha}  & \textbf{Onde se lê}  & \textbf{Leia-se}  \\
%    \hline
%    1 & 10 & auto-conclavo & autoconclavo\\
%   \hline
%\end{tabular}
%\end{table}
%
%\end{errata}
% ---

% ---
% Inserir folha de aprovação
% ---

% Isto é um exemplo de Folha de aprovação, elemento obrigatório da NBR
% 14724/2011 (seção 4.2.1.3). Você pode utilizar este modelo até a aprovação
% do trabalho. Após isso, substitua todo o conteúdo deste arquivo por uma
% imagem da página assinada pela banca com o comando abaixo:
%
%IMPORTANTE
%\includepdf{folhadeaprovacao_final.pdf}


%
%\begin{folhadeaprovacao}
%
%\begin{center}
%    \textbf{\normalsize\imprimirautor}
%
%    \vspace*{\fill}\vspace*{\fill}
%    \begin{center}
%       {\normalsize\bfseries\MakeTextUppercase{\imprimirtitulo}}
%    \end{center}
%    \vspace*{\fill}
%    
%    \hspace{.45\textwidth}
%    \begin{minipage}{.5\textwidth}
%        \imprimirpreambulo
%        \SingleSpacing
%  {\normalsize\imprimirorientadorRotulo~\imprimirorientador\par}
%        \SingleSpacing
%        \SingleSpacing
%Data de aprovação: 14 de março de 2019.
%    \end{minipage}%
%    \vspace*{\fill}
%   \end{center}
%        
%
%
%   \assinatura{{\imprimirorientador} \\ Orientador} 
%   \assinatura{{Prof. Dr. Gilciano Saraiva Nogueira} \\ Avaliador PPGCF}
%   \assinatura{{Prof. Dr. Cristiano Christofaro Matosinhos} \\ Avaliador PPGED}
%   \assinatura{{Prof. Dr. Carlos Alberto Araújo Júnior} \\ Avaliador Externo à UFVJM e ao PPGCF}
%   \assinatura{{Prof. Dr. Alessandro Vivas Andrade} \\ Avaliador Externo ao PPGCF}
%   %\assinatura{\textbf{Professor} \\ Convidado 4}
%      
%   \begin{center}
%    \vspace*{0.5cm}
%     \textbf{\normalsize\imprimirlocal}
%     \par
%   \textbf{\normalsize\imprimirdata}
%    \vspace*{1cm}
%  \end{center}
%  
%\end{folhadeaprovacao}
% ---

% ---
% Dedicatória
% ---
%\begin{dedicatoria}
%   \vspace*{\fill}
%   \centering
%   \noindent
%   \textit{ Este trabalho é dedicado às crianças adultas que,\\
%   quando pequenas, sonharam em se tornar cientistas.} \vspace*{\fill}
%\end{dedicatoria}
% ---

% ---
% Agradecimentos
% ---

%\begin{resumo}[agradecimentos]
\begin{agradecimentos}
\setlength{\parskip}{\onelineskip}  % tente também \onelineskip


\setlength{\parskip}{0.cm}  % tente também \onelineskip
\end{agradecimentos}


% ---
% Epígrafe
% ---
\begin{epigrafe}
    \vspace*{\fill}
	\begin{flushright}
		\textit{``Epígrafe''\\
		(Matrix)}
	\end{flushright}
\end{epigrafe}
% ---

% ---
% RESUMOS
% ---

% resumo em português
%\setlength{\absparsep}{18pt} % ajusta o espaçamento dos parágrafos do resumo
\begin{resumo}[RESUMO]

\noindent O resumo segue a norma ABNT NBR 6028:2003. Texto redigido pelo autor, na voz ativa e na terceira pessoa do singular, com os pontos relevantes do trabalho. Deve informar ao leitor a  finalidade, a metodologia, o resultado e/ou as conclusões do trabalho, evitando-se símbolos, contrações, fórmulas, equações, diagramas, etc. Deve ser redigido em parágrafo único com, no mínimo 150 e no máximo 500 palavras. Deve-se usar o mesmo recurso tipográfico do texto. No título, caixa alta, negrito e centralizado; espaçamento 1,5 entre linhas e também entre o titulo da seção e o texto. Logo abaixo do resumo devem figurar as palavras-chave, indicadas pela expressão “Palavras-chave:” e as palavras iniciadas em caixa alta e separadas entre si e finalizadas por ponto final. Se utilizar mais de uma linha, a partir da segunda deve ser alinhada com o início da palavra-chave da primeira linha. As palavras-chave devem ser preferencialmente escolhidas em vocabulário controlado, ex.: Terminologia de Assuntos da  Biblioteca Nacional, disponível em <http://catalogos.bn.br/>. Quando o resumo está incluído no próprio documento não se insere a referência antes do resumo. O resumo em inglês (Abstract) é obrigatório, mas poderá também ser traduzido para tantos idiomas quantos forem necessários para a difusão do trabalho, usando-se os seguintes cabeçalhos: Résumé (Francês), Resumen (Espanhol) Zusammenfassung (Alemão) etc.

\vspace{\onelineskip}
\noindent \textbf{Palavras-chave}: Palavra 1. Palavra 2. Palavra palavra palavra 3. Palavra palavra palavra 4. Palavra 5.

\end{resumo}

% resumo em inglês
\begin{resumo}[ABSTRACT]
 \begin{otherlanguage*}{english}
   \noindent 

   \vspace{\onelineskip}

   \noindent Segue as mesma estrutura mancionada no item resumo. Text text text text text text text text text text text text text text text text text text text text text text text text text text text text text text text text text text text text text text text text text text text text text text text text text text text text text text text text text text text text text text text text text text text text text text text text text text text text text text text text text text text text text text text text text text text text text text text text text text text text text text text text text text text text text text text text text text text text text text text text text text text text text text text text text text text text text text text text text text text text text text text text text text text text text text text text text text text text text text text text text text text text text text text text text text text text text text text text text text text text text text text text text text text text text text text text text text text text text text text text text text.
   \textbf{Keywords}: Keyword 1. Keyword keyword keyword 2. Keyword keywordkeyword 3. Keyword 4. Keyword 5. 
 \end{otherlanguage*}
\end{resumo}
\vfill

% resumo em francês 
%\begin{resumo}[Résumé]
% \begin{otherlanguage*}{french}
%    Il s'agit d'un résumé en français.
% 
%   \textbf{Mots-clés}: latex. abntex. publication de textes.
% \end{otherlanguage*}
%\end{resumo}

% resumo em espanhol
%\begin{resumo}[Resumen]
% \begin{otherlanguage*}{spanish}
%   Este es el resumen en español.
%  
%   \textbf{Palabras clave}: latex. abntex. publicación de textos.
% \end{otherlanguage*}
%\end{resumo}
% ---

% ---
% inserir lista de ilustrações
% ---
\pdfbookmark[0]{\listfigurename}{lof}
\listoffigures*
\cleardoublepage
% ---

% ---
% inserir lista de tabelas
% ---
\pdfbookmark[0]{\listtablename}{lot}
\listoftables*
\cleardoublepage
% ---

% ---
% inserir lista de abreviaturas e siglas
% ---
\begin{siglas}
\label{Siglas}
\item[ALS] \textit{Airborne Laser Scanning}
%\item[$Alt$] Altura extraída \textit{pixel}
\item[AMF] Área de Manejo Florestal 
\item[$arv$] Referência a dados de árvore do inventário florestal

\end{siglas}

% ---

% ---
% inserir lista de símbolos
% ---
%\begin{simbolos}
%  \item[$DAP $] Diâmetro medido à 1,3$m$ do chão
%  \item[$ \li $] Light Detect and Range
%  \item[$ \zeta $] Letra grega minúscula zeta
%  \item[$ \in $] Pertence
%\end{simbolos}
% ---

% ---
% inserir o sumario
% ---
\pdfbookmark[0]{\contentsname}{toc}
%\tableofcontents*
\tableofcontents
\cleardoublepage
% ---



% ----------------------------------------------------------
% ELEMENTOS TEXTUAIS
% ----------------------------------------------------------
\textual

% ----------------------------------------------------------
% Introdução (exemplo de capítulo sem numeração, mas presente no Sumário)
% ----------------------------------------------------------
\chapter{Algumas orientações}
%\addcontentsline{toc}{chapter}{Algumas orientações}
% ----------------------------------------------------------


As seções e/ou subseções não utilizadas devem ser retiradas, incluindo as preliminares.

Após a retirada, deve alterado somente o número da primeira página textual. As demais serão alteradas automaticamente. Clicar no número, ir em “formatar número de páginas”, “iniciar em” e alterar.

O Sumário basta atualizar, e inserir um espaço emtre as seções primárias. Selecione o sumário por completo, “atualizar campo”, atualizar o índice inteiro.
As listas não são atualizadas automaticamente.

Para inserir títulos nas seções e subseções, basta substituir o texto, na formatação fornecida.

Os casos não abordados no modelo devem seguir o Manual de normalização: monografias, dissertações e teses, da UFVJM, conforme as normas da ABNT vigentes.

Os títulos  das seções devem ser alinhados à esquerda, obedecendo a numeração progressiva em algarismos arábicos separados por um espaço de caractere, até a seção quinária. 

Devem iniciar sempre no anverso da folha. Os indicativos de seções sem numeração devem ser centralizados.

\begin{enumerate}
\item Os títulos das seções, subseções e seu texto devem ser separados por um espaço de 1,5 entre linhas. Entre o texto e o título seguinte o espaçamento também deve ser de 1,5. Caso o título ocupe mais de uma linha, as seguintes devem obedecer ao espaçamento (1,5) da primeira. Devem-se adotar os seguintes procedimentos: utilizar algarismos arábicos sequenciais;
\item limitar a numeração até a seção quinária;
\item utilizar números inteiros a partir de 1 no indicativo das seções primárias;
\item inserir um texto em cada seção;
\item apontar as subseções com o indicativo da seção primária, seguido pelo número que lhe for atribuído, separado por ponto, o que  se recomenda para as demais seções;
\item não utilizar ponto, hífen, travessão, parêntese ou outro sinal entre o indicativo de seção e o titulo;
\item alinhar, abaixo da primeira letra da primeira palavra do título, a partir da segunda linha, títulos que ocupem mais de uma linha. 
\end{enumerate}

\begin{figure}[htb!]
	\centering
	\caption{Estrutura do trabalho.}
	\label{fig:fig1}
	\includegraphics[scale=.95]{img_trab/fig1}
	\Fonte{1.cm}{Elaborada pelo autor.}
\end{figure}


% ---
% Capitulo de revisão de literatura
% ---
%\chapter{Estado da arte}
% ---
%Para elaborar sua resposta, mostre através de uma pequena revisão de literatura se outros trabalhos tentam resolver a sua pergunta geradora e, em caso afirmativo, como o fazem. 
%

% ---



% ----------------------------------------------------------
% PARTE
% ----------------------------------------------------------
\part{Parte 1}
% ----------------------------------------------------------

% ---
\chapter*{Capítulo 1}
% ---



% ---
%\section{Metodologia}
% ---



% ----------------------------------------------------------
% PARTE
% ----------------------------------------------------------
%\part{Considerações finais}
% ----------------------------------------------------------

% ---
% primeiro capitulo de Resultados
% ---
\chapter*{Considerações finais}
\addcontentsline{toc}{chapter}{Considerações finais}
% ---




% ---
% segundo capitulo de Resultados
% ---




% ----------------------------------------------------------
% Finaliza a parte no bookmark do PDF
% para que se inicie o bookmark na raiz
% e adiciona espaço de parte no Sumário
% ----------------------------------------------------------
\phantompart

% ---
% Conclusão
% ---
%\chapter{Conclusão Geral}
% ---


% ----------------------------------------------------------
% ELEMENTOS PÓS-TEXTUAIS
% ----------------------------------------------------------
\postextual
% ----------------------------------------------------------

% ----------------------------------------------------------
% Referências bibliográficas
% ----------------------------------------------------------
\bibliography{tese}

% ----------------------------------------------------------
% Glossário
% ----------------------------------------------------------
%
% Consulte o manual da classe abntex2 para orientações sobre o glossário.
%
%\glossary

% ----------------------------------------------------------
% Apêndices
% ----------------------------------------------------------

% ---
% Inicia os apêndices
% ---
%\begin{apendicesenv}
%
% Imprime uma página indicando o início dos apêndices
%\partapendices
%
% ----------------------------------------------------------
%\chapter{Quisque libero justo}
% ----------------------------------------------------------
%
%\lipsum[50]
%
% ----------------------------------------------------------
%\chapter{Nullam elementum urna vel imperdiet sodales elit ipsum pharetra ligula
%ac pretium ante justo a nulla curabitur tristique arcu eu metus}
% ----------------------------------------------------------
%\lipsum[55-57]
%
%\end{apendicesenv}
% ---


% ----------------------------------------------------------
% Anexos
% ----------------------------------------------------------

% ---
% Inicia os anexos
% ---
%\begin{anexosenv}
%
% Imprime uma página indicando o início dos anexos
%\partanexos
%
% ---
%\chapter{Morbi ultrices rutrum lorem.}
% ---
%\lipsum[30]
%
% ---
%\chapter{Cras non urna sed feugiat cum sociis natoque penatibus et magnis dis
%parturient montes nascetur ridiculus mus}
% ---
%
%\lipsum[31]
%
% ---
%\chapter{Fusce facilisis lacinia dui}
% ---
%
%\lipsum[32]
%
%\end{anexosenv}

%---------------------------------------------------------------------
% INDICE REMISSIVO
%---------------------------------------------------------------------
\phantompart
\printindex
%---------------------------------------------------------------------

\newpage
\thispagestyle{empty}
\textit{ \ \     }
% Posicao da watermark da capa
\AddToShipoutPicture*{\put(0,0){\includegraphics{img/capa.eps}}}

\end{document}
